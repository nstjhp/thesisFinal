Flowering plants are abundant on Earth.
In the model dicot plant species, \emph{Arabidopsis thaliana}, multiple endogenous and exogenous signals converge to initiate a change from vegetative to reproductive growth in optimal environmental conditions.
Much genetic and experimental research has gone into elucidating the biological mechanisms controlling the floral transition.
However there has been little mathematical modelling of this process.

The aim of this thesis was to gain an understanding of the essential features and dynamic properties underlying this developmental phase change from a systems and computational biology perspective.
Combining mathematical modelling with experimental results a core regulatory network was defined with multiple feedback loops.
Simplified models inevitably miss finer details of the biological system yet they provide a route to understanding the overall system behaviour.
This reductionist path allowed a tractable genetic regulatory network to be investigated without large numbers of parameters.

Not overfitting to data and parameter inference are two current challenges in systems biology.
Treating all unknowns as a probability within the setting of Bayes' theorem as a statistical framework allows for a solution to both of these issues.
This thesis investigates the use of a contemporary Bayesian inference algorithm, nested sampling, for inference problems typically found in systems biology where the data are few and noisy.
Nested sampling simultaneously calculates the key term for model comparison and also produces parameter inferences allowing uncertainty in models and predictions to be robustly quantified.

Network models are developed that can accurately reproduce experimental leaf number data, show important properties of the floral transition such as the ability to filter environmental noise and provide a clue on spatial patterning of an Arabidopsis shoot apex.
Incorporating network knowledge into a plant breeding program is an exciting goal for future developments addressing global food security.