This thesis was typeset using \LaTeX\ which was originally developed by Leslie Lamport and based on Donald Knuth's pioneering \TeX\@.
The body text is Minion Pro, set 11/17 pt on an almost 25 pc measure and the caption font is Myriad Pro.
My favourite monospace font, Monaco, was used in the few places necessary.
\XeLaTeX\ was used as the compiler to enable easy access to these fonts with the \texttt{mathspec} package.

I tried to base as many typographical decisions on recommendations found in Bringhurst's classic text \emph{The Elements of Typographic Style v3.2}, 2004.
My use of margin figures was influenced by the books of Tufte (\emph{The Visual Display of Quantitative Information}, 2001) and MacKay (\emph{Information Theory, Inference, and Learning Algorithms}, 2003).

I used Ti\textit{k}Z for diagrams and the \texttt{R} package \texttt{ggplot2} (created by Hadley Wickham) for plots.
Hopefully these figures, to a great extent, follow the best guidelines for clarity of presentation and data-ink ratio.
I also tried to avoid the use of true black where I could in the figures.

Finally, as far as possible I chose to use a colour-blind friendly palette for as many figures as I could.
This choice was based on my strong belief that scientists can be very poor at presenting images in a clear way, especially to those with colour-blindness.
My preferred reading on this matter is \emph{``How to make figures and presentations that are friendly to colorblind people''} by Masataka Okabe and Kei Ito, 2008.
Their colour-blind friendly palette can be found at \url{http://jfly.iam.u-tokyo.ac.jp/color/}.
